\documentclass[12]{article}

% --- Packages ---
\usepackage{xcolor}
\usepackage{listings}
\usepackage{pifont}
\usepackage{colortbl}
\arrayrulecolor{gray!10}

% --new commands--
%\ding{33-127} for symbols


% ---Article meta data ---
\title{Strawberry Fields (SF)}
\author{Dr. S. Damodaraiah}
\date{\today}

\lstset{ 
	basicstyle=\ttfamily\small, 
	breaklines=true, 
	frame=single, 
	captionpos=b, 
	keywordstyle=\color{blue}, 
	stringstyle=\color{red}, 
	commentstyle=\color{green!60!black}, 
	showspaces=false, 
	showtabs=false, 
	tabsize=2,
	morekeywords={with, as} 
} 


\begin{document}
\maketitle
Great! You're diving into Strawberry Fields and PennyLane — two powerful Python libraries developed by Xanadu for simulating and programming quantum photonic systems and hybrid quantum-classical machine learning workflows.

Let me briefly introduce both, compare them, and show how they fit together.
	
\section*{\ding{97} Strawberry Fields (SF)}
\textbf{Purpose:} Quantum photonics simulation and programming using \textbf{continuous-variable (CV) quantum computing}.
\subsection*{Developed by:}	
Xanadu
\subsection*{Features:}
\begin{itemize}
	\item Built for CV quantum computing (qumodes instead of qubits).
	\item Uses Blackbird language for circuit definition.
	\item Can simulate \textbf{Gaussian and Fock backends} (classical simulators for photonic circuits).
	\item Supports \textbf{quantum optical gates, interferometers,} and \textbf{measurement operations} like \colorbox{gray!30}{\texttt{homodyne}}, \colorbox{gray!30}{\texttt{heterodyne}}, \colorbox{gray!30}{\texttt{Fock}}.
\end{itemize}
\newpage
\subsection*{Example: Simple quantum photonic circuit}

\begin{lstlisting}[language=python]
	python
	
	
	import strawberryfields as sf
	from strawberryfields.ops import Sgate, Dgate, MeasureFock
	eng = sf.Engine("fock", backend_options={"cutoff_dim": 5})
	prog = sf.Program(1)
	
	with prog.context as q:
	Dgate(0.5) | q[0]
	Sgate(0.3) | q[0]
	MeasureFock() | q[0]
	
	result = eng.run(prog)
	print(result.samples)
\end{lstlisting}

\section*{\ding{97} PennyLane}
\subsubsection*{Purpose:} Framework for \textbf{hybrid quantum-classical computing}, especially for \textbf{machine learning}, compatible with many backends.
 \subsection*{\ding{98} Developed by:}
 Also by Xanadu
 \subsection*{\ding{68} Features:}
 \begin{itemize}
 	\item Focused on \textbf{quantum differentiable programming}.
 	\item Integrates well with \textbf{PyTorch, TensorFlow, JAX}.
 	\item Supports both \textbf{qubit and continuous-variable models}.
 	\item Can \textbf{interface with Strawberry Fields}, as well as IBM Qiskit, Rigetti Forest, etc.
 \end{itemize}
\subsection*{\ding{99} Used for:}
\begin{itemize}
	\item Variational Quantum Circuits (VQCs)
	\item Quantum Neural Networks (QNNs)
	\item Quantum optimization \& quantum chemistry
	\item Quantum machine learning
\end{itemize}	

\section*{\ding{101} Using Strawberry Fields inside PennyLane}
PennyLane has a plugin for Strawberry Fields, so you can define a photonic circuit and optimize it with gradient descent!
\subsubsection*{\ding{65}Example: Photonic circuit with PennyLane}	
\begin{lstlisting}[language=python]
	python
	
	
	import pennylane as qml
	from pennylane import numpy as np
	
	dev = qml.device("strawberryfields.fock", wires=1, cutoff_dim=10)
	
	@qml.qnode(dev)
	def circuit(x):
	qml.Displacement(x, 0.0, wires=0)
	qml.Squeezing(0.1, 0.0, wires=0)
	return qml.expval(qml.NumberOperator(0))
	
	x = np.array(0.1, requires_grad=True)
	print(circuit(x))  # Output expectation value
	
	grad = qml.grad(circuit)(x)
	print(grad)        # Gradient
	
\end{lstlisting}

\subsection*{\ding{39} Summary Table}
	\begin{table}[h!]
	\centering
	\begin{tabular} {l|l|l}
		\hline
		\textbf{Feature} & \textbf{Strawberry Fields} & \textbf{PennyLane} \\
		\hline
		
		Type & Photonic circuit simulator &	Hybrid QML framework \\
		\hline
		Focus	& CV quantum computing &	Differentiable programming (QML) \\
		\hline
		Supports qubits? &	No &	Yes\\
		\hline
		Supports CV (qumodes)? &	Yes &	Yes\\
		\hline
		Integration & 	Native &	Via plugin (e.g. \colorbox{gray!30}{\texttt{strawberryfields.fock)}}\\
		\hline
	\end{tabular}	
	\end{table}










\end{document}






